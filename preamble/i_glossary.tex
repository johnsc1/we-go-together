\unnumberedchapter{Interdisciplinary Glossary} 
\chapter*{Interdisciplinary Glossary}

% Break up this table into several ones if it takes up more than one page
\begin{center}
\begin{longtable}{r p{0.58 \textwidth}}

Check in & To follow the full add/commit/push process using Git in order to add new information or make changes to the repository on Github. In the case of open source projects or projects where pull requests are used to add more information to the main branch, the check in process should include a pull request at the end. The check in process includes using "git add", "git commit", and "git push" at the minimum. \\

Check out & To use "git pull" on the main branch of the repository to pull the most updated version of the repository. \\

Unchecked Resources & Resources at some stage of the check in process or which are have not started through that process, and have not completed the full check in process. For example, this includes information which "git add" has been applied to, but there was no commit created, or a commit which has been created but note yet pushed to Github. \\

Commit-Absent Copies & Local copies of the repository which are behind in the commit history compared to the Github repository, because there has been new commits pushed to Github/merged into the checked out branch and "git pull" has not been used in the CLI of the local repository to update it to include these commits \\

Github Circulation\\ Network (GCN) & The understanding of contributors, Git, and Github collectively as a centralized Peer-to-Peer network. The contributors act as the peer nodes, and the Github repository as the center of the system. Git connects the peer nodes to the center of the system through the CLI commands it supports for moving information, which is facilitated independently by the peers. \\

\end{longtable}
\end{center}

\begin{center}
\begin{longtable}{r p{0.58 \textwidth}}

Peer Repository & The copy of the given repository on the contributor's local machine.\\

Remote Central Repository & Specifically the main branch of the project repository on Github. \\

Content Library & Library objects which are detached from each other and held within a shell structure. These individual entities take on qualities of the library, acting both independently of and occasionally affected by the shell.  \\ 

Shell Library & Used to characterize a platform, such as Github, which had inherent qualities consistent with those of a library but functions in a decentralized structure where the libraries which make up the whole system are detached from each other and the shell, but are held within the shell. \\

Community Code\\Archival Librarian (CCAL) & A methodology of understanding who carries out the actions of both a librarian and archivist in cases involving software and FOSS communities. This term may be used to describe the collective action of multiple individuals or entities. An elaboration of the actions which characterize the CCAL can be found in the Integration Chapter under the Interdisciplinary Glossary section, specifically  \\

\end{longtable}
\end{center}