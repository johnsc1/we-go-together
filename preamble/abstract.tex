\unnumberedchapter{Abstract} 
\chapter*{Abstract} 

Github is an essential platform for Free Open Source Software, as it allows communities to collaborate on software projects. Because it is possible to apply humanities concepts to databases for a greater understanding of them, it is valid to approach Github this way. Interest sparked by Ochigame’s work to understand the librarian in library circulation is applied here by taking a set of concepts which includes fundamental library science concepts, research on coding as a literacy, and qualities of peer-to-peer networks to understand how Github qualifies as a library entity, how the circulation of information on Github is peer-to-peer and has library qualities, and the development of new interdisciplinary terminology to support these connections. This methodology is established as a valid approach to Github repositories through an in-depth case study on the pandas repository, using the interdisciplinary terms in the discussion to validate their practical use, including the Github Circulation Network and the Community Code Archivist Librarian. Implications for Library Science and Computer Science as disciplines are discussed along with interviews from scholars in Library and Information Science. Overall, this research implies that the case study methodology is a valid approach to understanding Github repositories, Github should be included as a database in the library catalog, libraries should further support literacy of code and archiving code, and the use of DOIs on Github and in the library should be more concretely established. 
