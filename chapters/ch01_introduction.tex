\chapter{Introduction}
\label{ch:intro}

This research begins with Github, which is widely used to store information, whether this is the best practice of pushing code to the platform for storage backup or curated repositories that are designed to organize some specific information \cite{wu2017github}. Within Github, written code and related information seems to circulate at both a surface and abstracted level. At the surface, code circulates through the use of a tool as dependencies supporting the functionality of other projects. On a deeper level, the circulation of code between local machines and Github is facilitated by the tool Git. Recent research explores new approaches to databases that include Critical Librarianship \cite{ackermans2020appeal}. Although the scope and abilities of Github are far wider than the standard database, this work implies that it may be possible to approach other structures which hold information from a perspective focused in Library Science. Because of this, it could be possible to extract further understanding of how Github functions by applying core concepts from both Library Science and Computer Science collectively. This is the first overall aim of the project. If Github does qualify as a Library entity, this implies that Github should belong in the Library catalog. The second aim of this project is to argue for this and explore why the library as a physical building is most qualified for this task. Lastly, this research will support argument for a new framework to approach Computer Science which reaches beyond the standard technical approach. 


\section{Motivation} 
\label{sec:motivation}

\subsection{Future Scholarly Work}

The motivation for this research begins with this being the basis for future scholarly work at the graduate level. Prior to the thesis, I worked on collaborative research in critical librarianship of databases. This research is motivated by learning more about Library Science as a discipline prior to beginning graduate studies in Library and Information Science in Fall 2022. Allegheny College does not have a Library Science curriculum, and there is no Library Science class as part of any existing curriculum. There is no Digital Humanities class being offered before I graduate, which could have given some insight into library science and librarianship from that perspective. By including Library Science as a major part of this research and conceptually linking this to prior knowledge in Computer Science, the foundation for future scholarly work is created. 

\subsection{Absence of Research on Library Science and Github}

Additionally, there is a significant lack of research on the relationship between Library Science and Github, and research which includes both Computer Science and Github conceptually. The existing research seems to focus on the technical application of Computer Science for the purpose of addressing issues in Library Science, such as building library databases. The closest research to the application of Library Science concepts to Computer Science is the work of Wu et. al in understanding how Github users curate repositories to
have information about a specific topic\cite{wu2017github}. Although the most common practice is to use Github repositories to hold code, there are repositories designed to hold and organize information about some specific subject. An example of a Github repository of curated information would be awesome-text-summarization. This repository holds information on types of Artificial Intelligence text summarization, written on the README file\cite{awesomesum}. There are images with computational details of the algorithms and links to scholarly articles about the concepts and tools they can be implemented with. Although the act of curation can be included in library science, Wu et al's research is not conceptually grounded in traditional library science literature. Their research is more focused on informatics, whereas the foundation of the thesis will be more focused on Library Science specifically through discussing and applying concepts from Library Science textbooks and Gorman's values. 

\subsection{Literacy of Code}

A vital part of this research addressing the question "Why does this research matter?". Vee's work in Coding Literacy helps to answer this question,  "Letting the needs of CS or software engineering determine the way we value coding more generally can crowd out other uses of code...allowing approaches to coding from the arts and humanities can make more room for these uses" \cite{vee2017coding}. At the surface, this limited approach to coding could limit what is able to be created with code. As a student in Computer Science, encouraging others who are not part of the discipline to take introductory classes is frequently met with protests that the approach to learning coding would not be understood by them. By including new approaches to coding that are not solely dependent upon a technical understanding, the practice of coding and literacy of that code is opened to individuals who are beyond Computer Science and software engineering.

\subsubsection{Consequences of Limited Literacy}

There are much deeper consequences associated with a limited approach, "Beyond precluding alternative values, this collapse of programming with CS or software engineering can shut entire populations out. In paradigms such as Atwood's, programming is limited to the types of people already welcome in its established professional context. This is clearly a problem for any hopes of programming abilities to become distributed more widely. CS as a discipline and programming as a profession have struggled to accommodate certain groups, especially women and people of color. In other words, historically disadvantaged groups in the domain of literacy are also finding themselves disadvantaged in programming" \cite{vee2017coding}. This limited perspective of programming Vee details becomes a detrimental issue when considered in terms of how technology has become a central part of society today. If individuals lack support in beginning to understand or access programming, they are at risk of not fully understanding parts of society that are deeply technological. When understood in terms of literacy then, women and people of color are at more risk of being further marginalized in society. Everyone who is part of this society where the influence of technology is exponentially increasing should be empowered to learn about code, not only learn how to code, as Vee argues\cite{vee2017coding}. Due to the nature of the issues presented by code illiteracy, there is a moral good implied in addressing these issues. They should also be able to have access to code beyond the internet, so if they do not have prior knowledge of how to find this information, they are able to be directed to that information. The library, being a central of information and including librarians who are able to guide patrons to the information they are seeking, is already equipped to provide this support. Libraries often include computer labs that allow free access to the internet, where Github could be accessed, and librarians can assist the user in the process of locating specific information. Even if the user does not know exactly what information they are looking for, the librarian's knowledge of locating information and resources in the library catalog will be able to point the user in the right direction. 

The reasoning for including Github alongside databases in the library is supported through first conceptual discussion of how library-related concepts and the Peer-to-Peer Model apply to Github. This is further supported by the application of the Interdisciplinary Glossary and terms from the Concepts section to discussion of how the Pandas repository on Github functions as a library entity. If Github can be approached as a library entity, it is implied that Github belongs as a resource in the library catalog and supported by the various services which libraries give.


\section{Current State of the Art}
\label{sec:stateofart}

The State of the Art as it applies to this research focuses mainly on the interdisciplinary research that exists in the space between Library Science and Computer Science. This research includes Marino's \textit{Critical Code Studies}, Vee's \textit{Coding Literacy}, and Wu et. al's \textit{Appropriation of Github for Curation}.  

Marino focuses on a humanities approach to extracting meaning from software, "Code is a social text, the meaning of which develops and transforms as additional readers encounter it over time and as contexts changes" \cite{marino2020critical}. Additionally, "the meaning of code is contingent upon and subject to the rhetorical triad of speaker, audience (both human and machine), and message" \cite{marino2020critical}. In the following chapter, Marino argues for an understanding of "code as text". If the more granular aspect of a Github repository that is code can be approached as a text, like a book, then it is reasonable to consider how the larger system of Github and the repository might be considered as a library. Marino's work is also of importance in arguing for the application of humanities concepts to code, which is traditionall viewed from a purely technical approach. If it is possible to understand the importnace of code beyond a technical lense, then this may be true for larger scale systems like Git and Github. 

In \textit{Coding Literacy}, Vee researches how coding now qualifies as a literacy and argues connects coding to the act of writing, "Computer programming also appears to have paralells to writing that go beyond its rhetorical framing: it is a socially situated, symbolic symbol that enables new kinds of expression as well as the scaling up of preexisting forms of communication. Like writing, programming has become a fundamental tool and method to organize information" \cite{vee2017coding}. Vee's work validates the application of humanities concepts to a technical practice, which creates a foundation for applying library science concepts to the actions of contributors and maintainers in FOSS communities. 

In \textit{The Appropriation of Github for Curation}, Wu et. al research how Github repositories can be used to hold specific information rather than software projects, "a new category of practice has recently emerged—software developers have begun appropriating GitHub repositories to create public resources lists. In 2014 and 2015, GitHub repositories, such as awesome-python and awesome-go, which curate resources about programming topics, gained vast popularity on GitHub" \cite{wu2017github}. This approach to utilizing Github for purposes beyond archiving software, and also archiving information, supports use of Github outside of Computer Science. 

\section{Goals of the Project}
\label{sec:goals}

\subsection{Building an Interdisciplinary Glossary}

The goals of this project are first to build an interdisciplinary dictionary which consists of new terminology. The purpose of this glossary is to provide some terminology to support interdisciplinary discussion of Computer Science and Library Science. The terms will exist between Library Science and the technical Computer Science terminology where existing terms to do accurately define the interactions between disciplines. These terms were influenced first by the American Library Association Glossary of Library and Information Science. This is where terminology relative to the field of Library Science has been defined. The fourth edition of the ALA Glossary has been consulted because this is the most recent edition of the glossary, published in 2013. The ALA glossary was first published in 1983, and new editions over time have updated definitions to move advancements in technology. Furthermore, the interdisciplinary glossary was influenced by the discussion of the core concepts defined in the Concepts chapter and how these apply to Github. Then, the interdisciplinary terminology will be applied in the discussion of how Github functions as a library using the Pandas repository the example. 

\subsection{Github in the Library}

The second goal of this research is to argue for the inclusion of Github and written code as resources in the library catalog. If Github qualifies as a library object, then it is necessary for Github to be included as a database in the library catalog alongside other databases, such as Worldcat. An key piece of this argument is Vee's argument for the literacy of coding. Much of her research focuses on how a variety of organizations, schools, and colleges have contributed to this literacy \cite{vee2017coding}. However, libraries are absent from Vee's discussion. A central issue to the literacy of code which Vee surfaces is the historical role of formal education, both K-12 and at the college level, and how an emphasis only on formal education can be harmful to those who do not have access to it. Vee then references several independent organizations and movements, such as Richard Stallman's work for Free Open Source Software\cite{vee2017coding}. Even with these organizations and movements, barriers remain. The individual in question must know of the existence of code and how to seek out these resources. Furthermore, what if they lack the means to access the information they are seeking? The library, including the physical library building with the resources held within, are uniquely suited to address this problem. Because Github and code are electronic documents, it is most logical for them to be part of the virtual library catalog. With this, the problem cannot be addressed without the physical library, which is supported by Gorman's argument for the preservation of the physical library. This is because the physical libraries holds computers and includes the librarian to address barriers in access to information. 

\subsection{Github as a Library Entity}

The third goal of the research is to argue that Github qualifies as a library entity. If Github does qualify as a library entity, this creates a new approach to understanding how Github functions beyond Github's technical functionality. It allows for a detailed understanding of how the community which builds the software functions, and understanding these communities better could create opportunities to understand how the communities can best function in general, and how they can fully utilize the Github repositories for project development and give meaning to those actions. Understanding how Github functions as a library entity will be studied in the Concepts chapter through the application of core library values\cite{gorman2000values} \cite{rubin2016foundationslis}, values of archivists\cite{rubin2016foundationslis}, and the actions of the librarian\cite{gorman2000values}. Then, to validate this method of application, a case study will be conducted with the pandas repository where the concepts will be discussed in how they apply to each tab of the Github repository. 


\section{Thesis Outline}
\label{sec:outline}

\subsection{Chapters Retained}

The structure of this thesis document has been modified to best support the interdisciplinary nature of the thesis research. The order of the chapters are Introduction, Related Works, Concepts, Integration, and Implications. Some of the chapters in the traditional structure have been kept, while others have been renamed to better describe their content and purpose or added to support additional content specific to this research. The chapters retained from the traditional structure are the Introduction and Related Works. These chapters are all essential to the overall flow and understanding of the thesis document. The Introduction and Conclusion are both standard across academic writing. Related Works is necessary to fully cover the literature review and references which support the research. The title Related Works is kept for this research to best cover details from all resources which support this research. This chapter includes sections for each resource, with additional discussion of how the resources interact being expanded upon in the following Concepts chapter.

\subsection{Chapters Added}

The Concepts chapter has been added to the thesis structure to highlight the core concepts that are the foundation for this research. The sources from which the concepts have been found are covered in the Related Works chapter, then the concepts are elaborated upon in the Concepts chapter. It is necessary to have this separation to best communicate the importance of these concepts to the research and keep the Related Works chapter at a length that is easier to navigate. The separation creates the space to discuss the concepts individually, how the concepts interact with others, and give some basis of how these concepts may or may not apply to the research. The overall concepts and sections are The Library, The Librarian, Literacy, Circulation, and Peer-to-Peer Networks. 

\subsection{Chapters Renamed}

Two chapters from the traditional thesis structure have been renamed to better represent their interdisciplinary nature. Firstly, the Methods section has been renamed Integration. Because this research must include the discussion of how core concepts from two disciplines apply to both, it is necessary to discuss how these disciplines might be integrated conceptually. This new chapter title better describes the function of this chapter for an interdisciplinary audience. The Integration chapter begins with explanation and further discussion of how the Interdisciplinary Glossary was formed. This is followed by application of this new terminology and discussion from the Concepts chapter to how the Pandas Github repository is an example of how Github functions as a library entity. The final section of the Integration chapter explains the planning process for the interviews with scholars, the process of forming the discussion questions, and presents the discussion questions. 

Lastly, the Experiments section has been renamed Implications. This section will begin with discussing the outcomes of the interviews with scholars to incorporate those ideas into the discussion of the implications of the thesis work. The chapter will be split into sections for discussing the outcomes of the case study in Integration, the interviews, and the implications for Computer Science and for Library Science. In both of these sections, the implications will be discussed at a disciplinary level and in practical applications. The Implications chapter has also absorbed qualities of the conclusion chapter to keep the conclusion concise, such as discussing consequences of the work, summarizing the results of the thesis, and presenting future work. 
