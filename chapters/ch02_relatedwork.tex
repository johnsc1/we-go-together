\chapter{Related Work} 
\label{ch:relatedwork}

The related works for this thesis have been organized into five sections in this order: Digital Humanities, Library Foundations, Changes in the Library, Computer Science Foundations, and Github. This is followed by Library Science works, first with foundation sources, then sources which reflect a shift in libraries and technology became more essential. The term "foundations" means in regards to the thesis, not necessarily the discipline (although it may be). Computer Science follows this, with Computer Science foundations, then Github. 

\section{Digital Humanities}

\subsection{Appealing to Your Better Judgement}

This thesis work begins with \textit{Appealing to Your Better Judgement: A Cal for Database Criticism}, where Ackermans, a scholar in the Digital Humanities, presented a new methods for approaching databases \cite{ackermans2020appeal}. They take hermeneutics and compare both distant and close reading to understand how a database might be read \cite{ackermans2020appeal}. Overall, Ackerman's article in relation this the thesis work functions as an example of how critical theory and humanities theories can be applied to technical structures to draw further meaning beyond the structure/code itself. If databases can be understood by applying humanities theory, including applying research on libraries, then it is valid to approach Github from a humanities perspective and one which includes library science. Although Github is not exactly a database, the primary and most simplified use of Github is to store, organize, and allow recurring access to that data. Therefore, applying library science concepts to Github is a valid approach. 

\subsection{Critical Code Studies and Coding Literacy}

Both of these texts, Marino's \textit{Critical Code Studies} and Vee's \textit{Coding Literacy} are part of the same Software Studies series. \textit{Critical Code Studies} written by Marino, who is a professor at the University of Southern California, will be considered first. A unique aspect of this book is that written lines of code, although with line-by-line walkthroughs are included in the book \cite{marino2020critical}. In this book, Marino, argues that there must be a new method of understanding code due to its significance now in culture, and instances where it has been taken out of context, specifically citing an incident with climate data \cite{marino2020critical}. Though not necessarily a book on learning to code, the printed code on physical library book and the line-by-line guide, plus the removal of immediate functionality of the code, supports Marino's argument of approaching code in a way which focuses on the underlying meaning of it \cite{marino2020critical}. 

In \textit{Coding Literacy}, Vee, professor and director of Composition at University of Pittsburgh, explores the history of literacy efforts, both for reading and writing, and for computer programming \cite{vee2017coding}. The main argument of the book is that what is considered literacy has shifted, as a result of how our daily communication occurs over platforms made of code and the demand for programming in the workplace, and now coding may be considered a literacy \cite{vee2017coding}. Vee also covers several code literacy efforts which are currently active, such as Black Girls Code \cite{vee2017coding}. Because coding is considered a literacy, and librarians and the library have a responsibility to carry out literacy efforts according to Rubin's value of Reading and the Book\cite{rubin2016foundationslis}, libraries are implicated in providing programming and resources to patrons on the literacy of code. This thesis will argue that such resources include having Github in the library catalog, how the traditional work of the librarian enables them to be prepared to support this implicated work, and how individual repositories might be included in the library catalog. \textit{Coding Literacy} will be further elaborated upon in the Concepts chapter. 

\section{Library Foundations}

\subsection{Our Enduring Values}

Gorman is a scholar whose work in building a concise foundation for the discipline of Library Science was first recommended by a research librarian on Tuesday September 14th, 2021, in an orientation to the field of Library Science. \textit{Our Enduring Values} is a text used in multiple introductory Library and Information Science courses across different universities. Based on limited existing theoretical foundations of library science, Gorman seeks to define and argue for a more clear foundation for the discipline, including several core values, actions of the librarian, and beginning to address the complications of the emerging importance of technology and how this may complicate the work of libraries \cite{gorman2000values}. In this book, Gorman argues for the preservation of the physical library through the emergence of technology (especially preserving the physical books in the library collection, rather than them being replaced with e-books), questions the integrity of online resources, and discusses what would be the impossible task of virtually preserving all library resources, as some could not be accurately replicated in a virtual format. The inclusion of Gorman's foundational disciplinary work in this thesis helps to address the lack in prior library science knowledge mentioning in the introduction. Gorman's emphasis on preserving the library's core purposes through the disciplinary changes are of importance to this thesis work in understanding the integration of disciplines in a way which would be constructive for library science as a discipline. 

\subsection{ALA Glossary}

The ALA Glossary of Library and Information Science is a glossary of terms specific to the discipline, with the most recent version available having been published in 2013 \cite{glossary2013}. This resource aided in  understanding the specifics of how LIS defines certain terminology, such as the specifics of curation, or if terms are less defined, such as circulation. In comparison with the ALA glossary first released in 1983, it is possible to see the shift in terminology definitions as a result of changing technology over the 30 year period between versions \cite{glossary1983}. Specifically the definition of the library will be applied in the Concepts chapter as one of the library definitions to understand how Github might function as a library entity, prior to discussion of how Gorman's values may apply to Github. 

\subsection{Foundations of Library and Information Science}

Published roughly sixteen years after \textit{Our Enduring Values}, \textit{Foundation of Library Science and Information Science: Fourth Edition} is where Rubin defines their own core values of the LIS discipline, creating at a point where Library Science now includes Information Science and technology has become more essential through daily life \cite{rubin2016foundationslis}. Like \textit{Our Enduring Values}, this textbook is used as a reading in several introductory LIS courses at different universities. Following the conclusion of Rubin's values, the Social of America Archivist values are listed and defined \cite{rubin2016foundationslis}. Because of the archival nature of Github, the SAA core values are also discussed in terms of Github and this thesis work in the Concepts chapter. Together, Rubin and Gorman's values for libraries , and the SAA core values and Gorman's actions of the librarians are discussed in how they do and do not apply to Github in the Concepts chapter, then are applied in a case study of a Github repository to establish and validate the methodology. 

\section{Changes in the Library}

\subsection{Our Enduring Values Revisited}

Around the same time \textit{Foundations of Library and Information Science: Fourth Edition} was published, Gorman published an updated version of \textit{Our Enduring Values Revisited: Librarianship in an Ever-changing World}. Here, he continues to argue for the preservation of the physical library in its current state, especially with the complications that arise from the reliance on technology, and the risks to integrity in an internet resource compared to a resource in the library catalog, whose validity has been confirmed \cite{gorman2000values}. Because of the importance of Gorman's work to the discipline, this version affirms the continued importance of how this thesis research should occur in a way which allows for implicated changes, but does so in a way with respect to the necessary core functions of the library and with respect to the integrity of information discussed. 

\subsection{Coding for Children and Adults in Libraries}

\textit{Coding for Children and Adults in Libraries} shows that there are established instructions for librarians and library staff \cite{harrop2018codinglibrary}. The existence of these instructions supports that there are resources available for libraries to become prepared for supporting code literacy efforts, and that these efforts support patrons of all ages. Because this work is already established, and because code can now be considered as a literacy due to Vee's research into what a literacy is and the overall importance of code \cite{vee2017coding}, supports that libraries have an opportunity to be involved in code literacy, and that this is both beginning to occur already and this effort may be expanded. 

\subsection{Meaningful Space in the Digital Age}

Malczewski, whose work in \textit{Meaningful Space in the Digital Age} is cited by Rubin in defining what the library is when Rubin is presenting their values \cite{malczewski2014meaningful} \cite{rubin2016foundationslis}. This definition begins with "A library building is more than a container for content, digital or print. It is a cultural space recognized as a place of learning" \cite{rubin2016foundationslis}. Defining the library, which is further elaborated upon in the Concepts chapter, is essential to this thesis for understanding how Github would qualify as a library entity. In addition to applying Gorman's values, it is necessary for concise definitions to be applied first and Github to be considered in terms of these definitions. 

\section{Computer Science Foundations}

\subsection{Peer-to-Peer}

Historically, the application of P2P networks in a Computer Science context started in the late 90s with the software Napster, explained by Fox in their paper published in IEEE in 2001 as "Napster lets any client advertise the MP3 files stored on its disk and download MP3 files from other clients connected to the Napster server network" \cite{fox2001peer}. Here, Napster is noted as having a server, which is a centralized element with a somewhat decentralized system. Within the following five years, Calvanese et. al, all faculty in Computer Science and Information Science, were exploring more decentralized and further specifically computational uses of P2P systems in \textit{Logical foundations of peer-to-peer data integration}, published in ACM in the mid-2000s \cite{calvanese2004logical}. Calvanese et. al also reference Napster's foundational position in P2P networks in software applications, "The P2P paradigm was made popular by Napster, which employed a centralized database with references to the information items (files) on the peers" \cite{calvanese2004logical}. 

Two books on the fundamentals of peer-to-peer networks were consulted to better understand what qualities a peer-to-peer system consists of so that these may be applied Github and Git. \textit{The Handbook of Peer-to-Peer Networking} includes multiple different papers on peer-to-peer systems, including a series of qualities which "most" peer-to-peer systems have \cite{peertopeerhandbook}. In the Concepts chapter, these qualities are discussed in terms of how they apply to Git Flow, from the local repository on a contributor's computer, through the Git process of uploading changes to Github, and then pulling them from Github to another contributor's repository. 

Although one of the peer-to-peer qualities is decentralized \cite{peertopeerhandbook}, \textit{Peer-to-Peer Computing: Principles and Applications} states that the centralization-to-decentralization of a peer-to-peer system exist on a spectrum between these two endpoints, and that it is possible for a centralized system to be peer-to-peer \cite{peertopeercomputing}. However, prior work in applying peer-to-peer networking to version control systems seems to be focused in making decentralized VCSs and studying them, rather than evaluating centralized ones. 

\subsection{The Cathedral and the Bazaar}

In this thesis, Computer Science is approached through open source spaces, being that Github is one, and through the communities surrounding FOSS projects. Because of this, it is necessary to understand the foundations of open source software and how these communities function. \textit{The Cathedral and the Bazaar} was written prior to the existence of the rich FOSS development community that is Github, but the qualities of the communities and projects discussed in the paper still apply to Github today. Raymond's work to define the qualities of FOSS communities is regarded as a key text in understanding how these communities function, and the differences between commercial software (cathedral) and FOSS community-built software (bazaar) \cite{raymond2001cathedral}. It is in this paper were Raymond asserts Linus's Law, "given enough eyes, all bugs are shallow", which essentially means that FOSS development communities should be populous communities for the best possible outcome for the project \cite{raymond2001cathedral}. The difference between the cathedral and the bazaar as a whole in this paper raises the centralized vs decentralized idea again, applying the decentralized to the FOSS community and therefore Github, and specifically to the decentralized action of community members who each take different action to build the software. Therefore this applies to the contributors, acting as peers. 

\section{Github}

\subsection{The Appropriation of Github for Curation}

In this article, Wu et. al explore the instances of Github users taking the repository template and using it to store information they curate, rather than the traditional use of software projects and FOSS development communities \cite{wu2017github}. This work supports that Github repositories have multiple purposes, including but not limited to building software. It supports that Github repositories are valid in their use of holding storing curated information. 

An example of a repository which has been utilized to store curated information is awesome-text-summarization, where maintainers icoxfog417 abhishekmamdapure have collected a variety of fundamental information about text summarization algorithms \cite{awesomesum}.

\subsection{Git and Github Docs}

An important piece of all software and platforms for programming is the documentation, which helps users and developers understand how the software functions. Documentation essentially functions as a guide, much like instructions for a board game would. The documentation for possible actions, functionality, and commands in Git and Github is compared with the core values of the library and SAA core values, as well as the actions of the librarian, to understand how both function in a library context. 

\subsection{pandas Github repository}

The pandas repository, which is the main repository for the pandas FOSS project, includes a highly active FOSS development community which seems to utilize a full extent of the Github repository, inclduing activity on every tab of the repository, following best practices such as discussing in issues before making a pull request, having contributing guidelines, creating a detailed README, and releasing software through the support release action on Github \cite{pandasrepo}. Pandas also has a DOI for the project, indictated at the top section of the README, whcih will be further discussed in the Implications chapter. Pandas is included in the thesis work as it will be the focus of the case study in the Integration chapter. 

\subsection{Github Archive Program}

Lastly, the Github Archive program is an excellent example of how Github is expanding into the process of physically archiving repositories from Github, using film, which is best for preservation according to the program \cite{githubarchiveboxes} \cite{arcticcodevault}. 

This shows how Github is equipped to handle the information on their platform, and in some ways sets a precedence of how to determine what is preserved. By archiving these electronic repositories in physical copies, Github is also engaging with the idea of LOCKSS, which is further emphasized through the decision for there to be three "greatest hits" archive boxes in different libraries on different continents \cite{arcticcodevault}. If libraries are going to take on the work of archiving FOSS communities, it will be important to consider how the Github Archive Program is already doing this work. 

